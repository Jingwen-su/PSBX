% Options for packages loaded elsewhere
\PassOptionsToPackage{unicode}{hyperref}
\PassOptionsToPackage{hyphens}{url}
%
\documentclass[
]{article}
\usepackage{lmodern}
\usepackage{amssymb,amsmath}
\usepackage{ifxetex,ifluatex}
\ifnum 0\ifxetex 1\fi\ifluatex 1\fi=0 % if pdftex
  \usepackage[T1]{fontenc}
  \usepackage[utf8]{inputenc}
  \usepackage{textcomp} % provide euro and other symbols
\else % if luatex or xetex
  \usepackage{unicode-math}
  \defaultfontfeatures{Scale=MatchLowercase}
  \defaultfontfeatures[\rmfamily]{Ligatures=TeX,Scale=1}
\fi
% Use upquote if available, for straight quotes in verbatim environments
\IfFileExists{upquote.sty}{\usepackage{upquote}}{}
\IfFileExists{microtype.sty}{% use microtype if available
  \usepackage[]{microtype}
  \UseMicrotypeSet[protrusion]{basicmath} % disable protrusion for tt fonts
}{}
\makeatletter
\@ifundefined{KOMAClassName}{% if non-KOMA class
  \IfFileExists{parskip.sty}{%
    \usepackage{parskip}
  }{% else
    \setlength{\parindent}{0pt}
    \setlength{\parskip}{6pt plus 2pt minus 1pt}}
}{% if KOMA class
  \KOMAoptions{parskip=half}}
\makeatother
\usepackage{xcolor}
\IfFileExists{xurl.sty}{\usepackage{xurl}}{} % add URL line breaks if available
\IfFileExists{bookmark.sty}{\usepackage{bookmark}}{\usepackage{hyperref}}
\hypersetup{
  pdftitle={Mon analyse R en mathématiques},
  pdfauthor={Jingwen SU},
  hidelinks,
  pdfcreator={LaTeX via pandoc}}
\urlstyle{same} % disable monospaced font for URLs
\usepackage[margin=1in]{geometry}
\usepackage{color}
\usepackage{fancyvrb}
\newcommand{\VerbBar}{|}
\newcommand{\VERB}{\Verb[commandchars=\\\{\}]}
\DefineVerbatimEnvironment{Highlighting}{Verbatim}{commandchars=\\\{\}}
% Add ',fontsize=\small' for more characters per line
\usepackage{framed}
\definecolor{shadecolor}{RGB}{248,248,248}
\newenvironment{Shaded}{\begin{snugshade}}{\end{snugshade}}
\newcommand{\AlertTok}[1]{\textcolor[rgb]{0.94,0.16,0.16}{#1}}
\newcommand{\AnnotationTok}[1]{\textcolor[rgb]{0.56,0.35,0.01}{\textbf{\textit{#1}}}}
\newcommand{\AttributeTok}[1]{\textcolor[rgb]{0.77,0.63,0.00}{#1}}
\newcommand{\BaseNTok}[1]{\textcolor[rgb]{0.00,0.00,0.81}{#1}}
\newcommand{\BuiltInTok}[1]{#1}
\newcommand{\CharTok}[1]{\textcolor[rgb]{0.31,0.60,0.02}{#1}}
\newcommand{\CommentTok}[1]{\textcolor[rgb]{0.56,0.35,0.01}{\textit{#1}}}
\newcommand{\CommentVarTok}[1]{\textcolor[rgb]{0.56,0.35,0.01}{\textbf{\textit{#1}}}}
\newcommand{\ConstantTok}[1]{\textcolor[rgb]{0.00,0.00,0.00}{#1}}
\newcommand{\ControlFlowTok}[1]{\textcolor[rgb]{0.13,0.29,0.53}{\textbf{#1}}}
\newcommand{\DataTypeTok}[1]{\textcolor[rgb]{0.13,0.29,0.53}{#1}}
\newcommand{\DecValTok}[1]{\textcolor[rgb]{0.00,0.00,0.81}{#1}}
\newcommand{\DocumentationTok}[1]{\textcolor[rgb]{0.56,0.35,0.01}{\textbf{\textit{#1}}}}
\newcommand{\ErrorTok}[1]{\textcolor[rgb]{0.64,0.00,0.00}{\textbf{#1}}}
\newcommand{\ExtensionTok}[1]{#1}
\newcommand{\FloatTok}[1]{\textcolor[rgb]{0.00,0.00,0.81}{#1}}
\newcommand{\FunctionTok}[1]{\textcolor[rgb]{0.00,0.00,0.00}{#1}}
\newcommand{\ImportTok}[1]{#1}
\newcommand{\InformationTok}[1]{\textcolor[rgb]{0.56,0.35,0.01}{\textbf{\textit{#1}}}}
\newcommand{\KeywordTok}[1]{\textcolor[rgb]{0.13,0.29,0.53}{\textbf{#1}}}
\newcommand{\NormalTok}[1]{#1}
\newcommand{\OperatorTok}[1]{\textcolor[rgb]{0.81,0.36,0.00}{\textbf{#1}}}
\newcommand{\OtherTok}[1]{\textcolor[rgb]{0.56,0.35,0.01}{#1}}
\newcommand{\PreprocessorTok}[1]{\textcolor[rgb]{0.56,0.35,0.01}{\textit{#1}}}
\newcommand{\RegionMarkerTok}[1]{#1}
\newcommand{\SpecialCharTok}[1]{\textcolor[rgb]{0.00,0.00,0.00}{#1}}
\newcommand{\SpecialStringTok}[1]{\textcolor[rgb]{0.31,0.60,0.02}{#1}}
\newcommand{\StringTok}[1]{\textcolor[rgb]{0.31,0.60,0.02}{#1}}
\newcommand{\VariableTok}[1]{\textcolor[rgb]{0.00,0.00,0.00}{#1}}
\newcommand{\VerbatimStringTok}[1]{\textcolor[rgb]{0.31,0.60,0.02}{#1}}
\newcommand{\WarningTok}[1]{\textcolor[rgb]{0.56,0.35,0.01}{\textbf{\textit{#1}}}}
\usepackage{graphicx,grffile}
\makeatletter
\def\maxwidth{\ifdim\Gin@nat@width>\linewidth\linewidth\else\Gin@nat@width\fi}
\def\maxheight{\ifdim\Gin@nat@height>\textheight\textheight\else\Gin@nat@height\fi}
\makeatother
% Scale images if necessary, so that they will not overflow the page
% margins by default, and it is still possible to overwrite the defaults
% using explicit options in \includegraphics[width, height, ...]{}
\setkeys{Gin}{width=\maxwidth,height=\maxheight,keepaspectratio}
% Set default figure placement to htbp
\makeatletter
\def\fps@figure{htbp}
\makeatother
\setlength{\emergencystretch}{3em} % prevent overfull lines
\providecommand{\tightlist}{%
  \setlength{\itemsep}{0pt}\setlength{\parskip}{0pt}}
\setcounter{secnumdepth}{-\maxdimen} % remove section numbering

\title{Mon analyse R en mathématiques}
\author{Jingwen SU}
\date{12/20/2020}

\begin{document}
\maketitle

\hypertarget{introduction}{%
\section{Introduction}\label{introduction}}

À la fin du cours, tous les étudiants ont téléchargé les documents
pertinents sur Github et partagé les résultats avec nous. Afin de
profiter pleinement de ces packages R sur les mathématiques, je vais
lire, analyser et évaluer les articles de mes collègues pour améliorer
mes lacunes.

Ceci est une introduction à l'article lu dans cet article. Si vous
souhaitez en savoir plus, vous pouvez rechercher des références.

\begin{itemize}
\tightlist
\item
  \textbf{Title:} pacma
\item
  \textbf{Auteurs:} Jingwen SU
\item
  \textbf{Lien sur Github:}
  \href{https://github.com/Jingwen-su/PSBX/blob/main/Document\%20final/gr01_Jingwen_SU_pacman.pdf}{Jingwen
  SU}
\end{itemize}

\hypertarget{synthuxe8se-du-travail-en-question}{%
\section{Synthèse du travail en
question}\label{synthuxe8se-du-travail-en-question}}

Ceci est un article d'introduction sur pacma. L'auteur a donné une brève
introduction sur le contenu de l'article au début. Dites au lecteur que
dans l'article, elle essaiera d'utiliser le package Pracma pour la
différence polynomiale, l'ajustement et l'ajustement linéaire.

Dans la deuxième partie, l'auteur a présenté le code d'installation du
package Pracma. Après cela, l'auteur a commencé à effectuer des calculs
de fonction.

Pour Différence et ajustement polynomial, il a introduit trois
fonctions: polyfit, polyfix et polyvalent. D'un certain point de vue, le
premier et le second sont similaires.

Pour Ajustement linéaire, l'auteur a d'abord passé au peigne fin le
processus d'ajustement linéaire en détail, puis a démontré en fonction
de chaque étape.

\hypertarget{contenu-principal-et-explication}{%
\section{Contenu principal et
explication}\label{contenu-principal-et-explication}}

\hypertarget{installer-le-package-pracma}{%
\subsection{1. Installer le package
pracma}\label{installer-le-package-pracma}}

\begin{Shaded}
\begin{Highlighting}[]
\CommentTok{#install.packages("parcma")}
\KeywordTok{library}\NormalTok{(pracma)}
\end{Highlighting}
\end{Shaded}

\hypertarget{les-fonctions}{%
\subsection{2. Les fonctions}\label{les-fonctions}}

\begin{itemize}
\item
  \textbf{polyfit (x, y, n)} génère des coefficients polynomiaux, et sa
  puissance est triée de haut en bas, et n \textless(longueur (x) -1)
  ajuste automatiquement les données.
\item
  \textbf{polyfix (x, y, n, xfix, yfix)} est également un paramètre de
  coefficient polynomial. Les paramètres xfix et yfix représentent les
  coordonnées du point de référence, ce qui signifie que la courbe
  d'ajustement doit passer ce point.
\item
  \textbf{polyvalent (p, x)} Selon le vecteur de coefficient polynomial
  P, générer un polynôme, puis calculer la valeur de la coordonnée x en
  fonction du polynôme.
\end{itemize}

\hypertarget{ajustement-linuxe9aire}{%
\subsection{3. Ajustement linéaire}\label{ajustement-linuxe9aire}}

Ici, nous devons maîtriser sa pensée logique. Dans un projet ou un
calcul, il n'y a souvent pas de package d'installation unique impliqué.
Nous devons apprendre à combiner plusieurs packages d'installation et à
maîtriser les fonctions fréquemment utilisées.

L'auteur raconte un processus complet de traitement des données. Pour
l'ajustement linéaire, lorsque nous obtenons les données:

\begin{itemize}
\item
  Le premier est le prétraitement des données (y compris l'élimination
  des valeurs aberrantes, la normalisation des données: suppression
  d'unités, soustraction de la valeur minimale, division par la valeur
  maximale, de sorte que les données soient dans l'intervalle (0, 1)).
\item
  Puis dessinez un nuage de points et établissez une relation
  fonctionnelle basée sur la distribution des points (pas nécessairement
  une fonction linéaire).
\item
  Ensuite, selon la relation de fonction, la relation de fonction
  linéaire est dérivée.
\item
  Selon la relation de fonction linéaire, effectuez un ajustement
  linéaire, résolvez le coefficient de corrélation, apportez la solution
  dans la fonction d'origine et dessinez l'image de la fonction.
\end{itemize}

La démonstration est la suivante:

\begin{Shaded}
\begin{Highlighting}[]
\KeywordTok{library}\NormalTok{(ggplot2)}

\KeywordTok{set.seed}\NormalTok{(}\DecValTok{11}\NormalTok{)}
\NormalTok{black <-}\StringTok{ }\ControlFlowTok{function}\NormalTok{(x) \{}
    \DecValTok{2} \OperatorTok{*}\StringTok{ }\NormalTok{x }\OperatorTok{*}\StringTok{ }\KeywordTok{exp}\NormalTok{(}\FloatTok{0.5} \OperatorTok{*}\StringTok{ }\NormalTok{x) }\OperatorTok{+}\StringTok{ }\KeywordTok{runif}\NormalTok{(}\KeywordTok{length}\NormalTok{(x), }\DataTypeTok{min =} \DecValTok{-1}\NormalTok{, }\DataTypeTok{max =} \DecValTok{1}\NormalTok{)}
\NormalTok{\}}
\NormalTok{x_test <-}\StringTok{ }\KeywordTok{seq}\NormalTok{(}\DecValTok{0}\NormalTok{, }\DecValTok{5}\NormalTok{, }\FloatTok{0.2}\NormalTok{)}
\NormalTok{y_test <-}\StringTok{ }\KeywordTok{black}\NormalTok{(x_test)}
 
\NormalTok{rigion <-}\StringTok{ }\KeywordTok{data.frame}\NormalTok{(}\DataTypeTok{x =}\NormalTok{ x_test, }\DataTypeTok{y =}\NormalTok{ y_test)  }\CommentTok{# Données prétraitées}
\KeywordTok{ggplot}\NormalTok{(rigion, }\KeywordTok{aes}\NormalTok{(x, y)) }\OperatorTok{+}\StringTok{ }\KeywordTok{geom_point}\NormalTok{(}\DataTypeTok{color =} \StringTok{"blue"}\NormalTok{)  }\CommentTok{# Dessinez un nuage de points à partir duquel vous pouvez construire un modèle de fonction exponentielle}
\end{Highlighting}
\end{Shaded}

\includegraphics{mon-R-en-mathematique_files/figure-latex/unnamed-chunk-2-1.pdf}

\begin{Shaded}
\begin{Highlighting}[]
\CommentTok{# Modèle de fonction:y = c1*t*exp(c2*t), Linéarisation des fonctions:lny = lnc1 + lnt + c2*t}
\CommentTok{# Substituer des variables et transformer des inconnues en coefficients de fonctions linéaires:lny -lnt = c2*t + k,}
\CommentTok{# La variable indépendante est t, la variable dépendante est lny-lnt et le coefficient est calculé par ajustement linéaire:k, c2}
\NormalTok{fun_y <-}\StringTok{ }\KeywordTok{log}\NormalTok{(y_test, }\DataTypeTok{base =} \KeywordTok{exp}\NormalTok{(}\DecValTok{1}\NormalTok{)) }\OperatorTok{-}\StringTok{ }\KeywordTok{log}\NormalTok{(x_test, }\DataTypeTok{base =} \KeywordTok{exp}\NormalTok{(}\DecValTok{1}\NormalTok{))}
\end{Highlighting}
\end{Shaded}

\begin{verbatim}
## Warning: NaNs produced
\end{verbatim}

\begin{Shaded}
\begin{Highlighting}[]
\NormalTok{fun_x <-}\StringTok{ }\NormalTok{x_test}
\NormalTok{relation <-}\StringTok{ }\KeywordTok{lm}\NormalTok{(fun_y }\OperatorTok{~}\StringTok{ }\NormalTok{fun_x)  }\CommentTok{# Établir une relation linéaire}
\KeywordTok{print}\NormalTok{(relation)  }\CommentTok{# Afficher le coefficient de relation, les résultat: c2 = 0.5583, k = 0.4780}
\end{Highlighting}
\end{Shaded}

\begin{verbatim}
## 
## Call:
## lm(formula = fun_y ~ fun_x)
## 
## Coefficients:
## (Intercept)        fun_x  
##      0.4780       0.5583
\end{verbatim}

\begin{Shaded}
\begin{Highlighting}[]
\NormalTok{k <-}\StringTok{ }\NormalTok{relation[[}\DecValTok{1}\NormalTok{]][}\DecValTok{1}\NormalTok{]}
\NormalTok{c2 <-}\StringTok{ }\NormalTok{relation[[}\DecValTok{1}\NormalTok{]][}\DecValTok{2}\NormalTok{]}
 
\CommentTok{# Apportez les valeurs de c2 et k pour trouverc1 = e^k}
\NormalTok{c1 <-}\StringTok{ }\KeywordTok{exp}\NormalTok{(k)}
\CommentTok{# Le modèle de fonction est:}
\NormalTok{fun_last <-}\StringTok{ }\ControlFlowTok{function}\NormalTok{(x) \{}
\NormalTok{    c1 }\OperatorTok{*}\StringTok{ }\NormalTok{x }\OperatorTok{*}\StringTok{ }\KeywordTok{exp}\NormalTok{(c2 }\OperatorTok{*}\StringTok{ }\NormalTok{x)}
\NormalTok{\}}
\CommentTok{# Dessinez le graphique ajusté}
 \KeywordTok{ggplot}\NormalTok{(rigion, }\KeywordTok{aes}\NormalTok{(x, y)) }\OperatorTok{+}\StringTok{ }\KeywordTok{geom_point}\NormalTok{(}\DataTypeTok{color =} \StringTok{"blue"}\NormalTok{, }\DataTypeTok{size =} \DecValTok{3}\NormalTok{) }\OperatorTok{+}\StringTok{ }\KeywordTok{stat_function}\NormalTok{(}\DataTypeTok{fun =}\NormalTok{ fun_last, }\DataTypeTok{color =} \StringTok{"red"}\NormalTok{, }\DataTypeTok{size =} \DecValTok{1}\NormalTok{)}
\end{Highlighting}
\end{Shaded}

\includegraphics{mon-R-en-mathematique_files/figure-latex/unnamed-chunk-2-2.pdf}

\hypertarget{evaluation-et-ruxe9sumer}{%
\section{Evaluation et résumer}\label{evaluation-et-ruxe9sumer}}

Selon mes normes, je pense que c'est un article moyen.

Tout d'abord, son format est très concis et il y a des explications
détaillées sur le contenu important. De plus, le contenu proposé a
également été démontré. Le cadre logique est très bon.

Cependant, il n'a pas donné une introduction générale au package
d'installation, et n'a pas donné de bibliographie, de sorte que les gens
ne pouvaient pas le comprendre davantage.

\end{document}
